\chapter{Reinforcement Learning}

A continuación, se va a explicar brevemente en qué consiste el Reinforcement Learning (a partir de ahora RL), además de explicar cómo se aplicará este area del aprendizaje automático a nuestro trabajo.

\section{RL}

El RL es un paradigma de aprendizaje automático que se centra en enseñar a un agente a tomar decisiones secuenciales para maximizar una recompensa acumulativa a lo largo del tiempo. En este enfoque de aprendizaje automático, el agente interactúa con un entorno dinámico y aprende a base de las recompensas y penalizaciones que conllevan sus decisiones.

\section{RL en Pokémon}

Como en este trabajo estamos llevando a cabo batallas pokemon, el agente tiene un rango de 4 opciones a la hora de realizar una accion. Estas opciones son los 4 movimientos que puede efectuar el pokemon que está en ese momento luchando. La forma en la que se entregarán recompensas y penalizaciones al agente será en base de si estos movimientos inflingen daño (quitan puntos de vida del pokemon al que se enfrenta), o si gana el combate. De esta forma el agente aprenderá qué movimientos son los óptimos para los distintos pokemons.

El agente también posee un estado, en relación con el entorno de este trabajo. Este estado incluirá valores como, por ejemplo, el tipo o tipos de Pokémon que el agente maneja y el Pokémon al que se enfrenta, las puntuaciones de vida de los Pokémon y los movimientos disponibles de los Pokémon.

Para elegir la accion que realizará el Pokémon, el modelo que se creará tendrá que recibir como inputs, todos los valores del estado comentados previamente y devolvera un output de 4 valores correspondientes a los 4 posibles movimientos del Pokémon controlado por el agente, de los cuales se elegirá el que tenga un mayor valor.