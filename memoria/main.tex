\documentclass{report}

% Packages
\usepackage[utf8]{inputenc}
\usepackage{amsmath}
\usepackage{amssymb}
\usepackage{amsthm}
\usepackage{graphicx}
\usepackage{float}
\usepackage{listings}
\usepackage{hyperref}
\usepackage{url}
\usepackage[strings]{underscore}

\hypersetup{
    colorlinks=true,
    linkcolor=blue,
    filecolor=magenta,      
    urlcolor=cyan,
    pdftitle={Overleaf Example},
    pdfpagemode=FullScreen,
}

\begin{document}

\begin{titlepage}
    \begin{center}
        \vspace*{1cm}
 
        \Large\textbf{Reinforcement Learning: Combates de Pokémon }
 
        \vspace{0.5cm}
	Antecedentes, implementación y resultados
             
        \vspace{1.5cm}
 
	    \textbf{Alejandro Jiménez, Alejandro Gómez, Alejandro Gómez ,Luis Crespo y Antonio Cabrera}
 
        \vfill
             
        Trabajo para el doble grado de\\
        Ingeniería del Software y Matemática Computacional\\
             
        \vspace{0.8cm}
      
        \includegraphics[width=0.4\textwidth]{figures/logo u-tad.png}
             
        Asignatura de topología\\
        U-tad\\
        España\\
        Enero 2024
             
    \end{center}
 \end{titlepage}

\tableofcontents
\listoffigures

\chapter{Introducción}

\section{¿Qué es un fractal?}



\end{document}
