\chapter{Introducción}

\section{Contexto}

Desde que Deep Blue gana a Kasparov en 1997, las máquinas han ido superando a los humanos en la mayoría de juegos clásicos. Con los avances en computación y en Deep Learning la diferencia entre los grandes maestros y los ordenadores es más grande que nunca. La nueva generación de bots para jugar a juegos hace uso del Reinforcement Learning, que consiste en entrenar a una red neuronal en base a lo que aprende jugando contra ella misma.

Pokémon es la franquicia de entretenimiento más grande del mundo, con más de 300 millones de juegos vendidos. El juego consiste en capturar criaturas llamadas Pokémon y entrenarlos para que luchen contra otros Pokémon. A pesar de que el juego fue pensado para niños, Pokémon cuenta con una escena competitiva muy profesional (el torneo mundial de 2024 contará con 2.000.000 de euros en premios), sin embargo el uso de inteligencias artificiales aun no se ha extendido como en otros juegos.

\section{Objetivos}

El objetivo de este trabajo es crear un bot que sea capaz de jugar a Pokémon de forma autónoma. Para ello se ha utilizado una versión simplificada del juego programada en python, en la que se eliminan ciertos elementos como el azar en la partida. El bot se ha entrenado utilizando el algoritmo de Reinforcement Q-Learning , haciendo uso de la librería de PyTorch.


